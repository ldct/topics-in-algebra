\listfiles
\documentclass{article}

\usepackage{amsmath}
\usepackage{amssymb}
\usepackage{mathtools}
\usepackage{titlesec}
\diagramstyle[labelstyle=\scriptstyle]

\DeclarePairedDelimiter\floor{\lfloor}{\rfloor}
\DeclareMathOperator{\Hom}{Hom}
\DeclareMathOperator{\Set}{Set}
\DeclareMathOperator{\MSet}{MSet}
\DeclareMathOperator{\End}{End}
\def\N{\mathbb{N}}
\def\R{\mathbb{R}}
\def\C{\mathbb{C}}
\def\Z{\mathbb{Z}}

\usepackage[a4paper,margin=1in]{geometry}

\setlength{\parindent}{0cm}
\setlength{\parskip}{1em}

\title{Topics in Algebra}
\date{}

\begin{document}
\maketitle

\section*{2.3.3}

If $G$ is a group such that $(ab)^2 = a^2b^2$ for all $a, b \in G$, show that $G$ must be abelian.

Proof: we have $aabb = abab$, hence $ab = ba$.

\section*{2.3.4}

If $G$ is a group in which $(ab)^i = a^ib^i$ for three consecutive integers $i$ for all $a, b \in G$, show that $G$ is abelian.

Proof: By cancelling $a$ and $b$ from the head and tail, we see that $a^ib^i = (ba)^i$ for there consecutive integers. We have $a^ib^i = (ba)^i$ and $a^{i+1}b^{i+1} = (ba)^{i+1}$. From the second equation, we have $a^{i+1}b^{i+1} = ba(ba)^i = baa^ib^i= ba^{i+1}b^i$. Cancelling, we have $a^{i+1}b = ba^{i+1}$. A similar proof shows that $a^{i+2}b = ba^{i+2}$. Hence we have $aba^{i+1} = ba^{i+2}$. Cancelling, we have $ab = ba$.

\section*{2.3.5}

Show that the conclusion of problem 4 does not follow if we assume the relation $(ab)^i = a^ib^i$ for just two consecutive integers.

Proof: Let $G$ be any nonabelian group, eg $S_3$. Then we have $(ab)^i = a^ib^i$ for $i = 0$ and $i = 1$.

\section*{2.3.6}

In $S_3$, give an example of two elements $x, y$ such that $(xy)^2 \ne x^2y^2$.

Solution: let $x = (1 2)$ and $y = (2 3)$. Then $x^2 = y^2 = e$. However $xy = (1 3 2)$ and $(xy)^2 = (1 2 3)$.

\section*{2.3.11}

If $G$ is a group of even order, prove that it has an element $a \ne e$ satisfying $a^2 = e$.

Proof: Let $x \sim y \iff xy = e$ and consider $G / \sim$. Equivalence classes have size 1 if it consists of an element whose square is $e$ and 2 otherwise. There must be an even number of equivalence classes with size 1. Since one of them is the class $\{e\}$, there must be one other class.

\section*{2.3.24}

Let $G$ be the set of all $2 \times 2$ matrices where $a, b, c, d$ are integers modulo 2, such that $ad - bc \ne 0$. Using matrix multiplication as the operation in $G$, prove that $G$ is a group of order 6.

By the properties of matrix multiplication and determinant, it is easy to see that $G$ is a group. The determinant of elements of $G$ must be 1. We count $G$ by partitioning it based on the number of nonzero entries, denoted $s(g)$. If $s(g) < 2$ then $|g| = 0$ since every term ($ad$ and $bc$) contains two factors, at least one of which is 0. For $s(g) = 2$, there cannot be one nonzero factor per term, so we get exactly 2 elements. For $s(g) = 3$, exactly one factor is zero, hence all 4 matrices belong to $G$. Finally for $s(g) = 4, |g| = 0$.


\end{document}
